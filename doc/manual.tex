\documentclass{article}

\newcommand{\thetitle}{The Julia Programming Language}

\usepackage{amsmath}
\usepackage[hyperfigures,bookmarks,bookmarksopen,bookmarksnumbered,colorlinks,linkcolor=black,citecolor=black,filecolor=blue,menucolor=black,pagecolor=blue,frenchlinks=true,pdftitle={\thetitle}]{hyperref}

\title{\thetitle}
\author{
Jeff Bezanson \vspace{0.5em}\\
Stefan Karpinski \vspace{0.5em}\\
Viral Shah \vspace{0.5em}
% ~~~
}

\begin{document}

\maketitle

Scientific computing has traditionally required the highest performance,
yet domain experts have largely moved to slower dynamic languages for
daily work. We believe there are many good reasons to prefer dynamic languages
for these applications, and we don't expect their use to diminish any time
soon. Fortunately, modern language design and compiler techniques make it
possible to mostly eliminate the performance trade-off and provide a
single environment productive enough for prototyping and performant enough
for deploying applications. However, an open-source language with these
characteristics has not emerged. Our project, Julia, fills this gap.

\end{document}
